%\documentclass[12p]{article}
%\usepackage{a4paper,makeidx}
%\usepackage[dvips]{graphics}
\documentclass{article}
\usepackage{graphicx}
\usepackage{subfig}
\usepackage{multirow}

%\documentclass[smallextended]{svjour3}
%\usepackage[dvips]{graphicx}
%\usepackage{subfig}
%%%%%%%%%%%%%%%%%%%%%%%%%%%%%%%%%%%%%%%%%%%%%%%%%%%%%%%%%%%%%%%%%%%%%%%%%%%%%%%%%%%%%%%%%%%%%%%%%%%%%%%%%%%%%%%%%%%%%%%%%%%%

\righthyphenmin=55
\newlength{\gnat}
\newlength{\figheight}
\newlength{\figwidth}
\setlength{\textwidth}{6in}
\setlength{\evensidemargin}{0.2in}
\setlength{\oddsidemargin}{0.2in}
\setlength{\topmargin}{0.0in}
\setlength{\textheight}{9in}
\setlength{\headsep}{10pt}
\setlength{\columnsep}{0.375in}

\newtheorem{theorem}{Theorem}
\newtheorem{proposition}{Proposition}
\newtheorem{lemma}{Lemma}
\newtheorem{corollary}{Corollary}
\newtheorem{definition}{Definition}
\newtheorem{remark}{Remark}
\newtheorem{claim}{Claim}

\def\IR{\Bbb R}
\def\tF{{\tilde F}}
\def\tG{{\tilde G}}
\def\tE{{\tilde E}}
\def\hF{{\hat F}}
\def\hG{{\hat G}}
\def\hE{{\hat E}}
\def\oF{{\overline{F}}}
\def\oG{{\overline{G}}}
\def\CC{{\cal C}}
\def\uF{{\underline{F}}}
\def\uG{{\underline{G}}}
\def\oC{{\overline{C}}}
\def\uC{{\underline{C}}}
\baselineskip= 20pt
%\input{tcilatex}

\begin{document}

\title {Random time transformation analysis of the Covid19 Pandemic 2020}

\author {
Nitay Alon
\\
{\em E-mail: \tt{nitayalon@mail.tau.ac.il}} \\
Isaac Meilijson
\\
{\em E-mail: \tt{isaco@tauex.tau.ac.il}} \\
{\em School of Mathematical Sciences} \\
{\em Raymond and Beverly Sackler Faculty of Exact Sciences} \\
{\em Tel-Aviv University, 6997801 Tel-Aviv, Israel} \\
}

%\titlerunning{Covid19 2020}
%\authorrunning{Alon and Meilijson}

\maketitle

\pagenumbering{arabic}

\begin{abstract}



%\keywords{
%\noindent{\bf JEL Classification Numbers:}
%
\end{abstract}

%\vspace{1cm}

%\noindent \hrulefill \hspace{12cm}

%\baselineskip= 20pt

%\newpage

\section{Introduction} \label{introduction}

\noindent {\bf The SIR epidemiological model}. Let $S(t) = K - X(t)$ be the number of susceptible cases at time $t$, expressed in terms of the number of affected cases $X(t)$ and the possibly unknown parameter $K$ (that may or may not be the population size $N$ usually substituted for $K$). Let $R(t)$ be the removed cases at time $t$, dead ($V(t)$) or recovered ($W(t)=R(t)-V(t)$). Let $I(t)=S(t)-R(t)$ be the number of infected cases at time $t$. The common formulation of the SIR (Susceptible, Infected, Removed) epidemiological model is the variant of the system of ODE
\begin{eqnarray}
dX(t) & = & \beta(t) g(I(t)) \max(0,1 - X(t)/K) dt \label{DEforX} \\
dR(t) & = & \gamma I(t) \label{DEforR} dt \\
I(t) & = & X(t)-R(t) \label{eqforI}
\end{eqnarray}
where $K=N$ and $g(x)=x$. The parameter $\beta(t)$ is estimated by smooth local regression of \linebreak $X$-increments with respect to the RHS of (\ref{DEforX}). The ratio ${{\beta(t)} \over \gamma}$ is of primary importance, as its transition from above to below $1$ indicates whether the epidemic is spreading or dwindling, whether the number of infected cases $I$ increases or decreases with time.

\bigskip

At the early stages of the epidemic, the substitution of $K$ by population size $N$ or anything big enough including $\infty$, has similar effects. This report will illustrate on the Covid19 2020 data that if $K$ is allowed to be a free parameter, the SIR system of equations above may accept an almost exact solution in which $\beta$ (and not only $\gamma$) is constant, and $K$ is the asymptote to which the number $X$ of affected cases converges, the maximal sub-population size to become affected, if the conditions prevalent at the monitoring stage stay in effect.

Equations (\ref{DEforR}) and (\ref{eqforI}) as well as the linear appearance of the susceptible cases in (\ref{DEforX}) are quite straightforward under a stationary regime, but the role of $I(t)$ in (\ref{DEforX}) needs some attention. The effective vicinity of a susceptible case need not be the entire infected cohort, just as the asymptote of affected cases need not be population size $N$.

Had the function $g$, modelling the impact of the current extent of infection on the emergence of new cases, been known, parameter estimation via these differential equations would have provided (unjustifiably) accurate estimates of the asymptote $K$. The moment $g(x)=x$ is rejected in favor of $g(x)=x^\alpha$ for some $\alpha<1$ as has been repeatedly suggested in the literature (see Bj{\o}rnstad, Finkenst\"{a}dt and Grenfell in their various publications, such as \cite{AAA} and the references therein), this shape parameter $\alpha$ has to be estimated together with $K$ (tacitly, the pair $(\beta, \gamma)$ is always estimated anew). If the estimates $\hat{K}$ and $\hat{\alpha}$ came out quite uncorrelated (empirically established or theoretically assessed by the Fisher information matrix), estimation of the parameter $K$ would have been insensitive to the choice of $\alpha$, that could then be determined once and for all as a compromise value between all countries. This not the case. $\hat{K}$ and $\hat{\alpha}$ are strongly negatively correlated and compensate for each other. Even if the model fits data admirably well {\em in sample} over a proper ridge of $(K,\alpha)$ values, the values of $K$ dramatically differ, and the standard error of $\hat{K}$ exceeds by orders of magnitude the values it would have had for fixed $\alpha$. Still, the Covid19 2020 indicates country by country that the measures taken from March to May 2020 induced rather stationary behavior, with $K$ values strongly significantly finite, not exceeding three times the number of affected cases in mid May 2020. The model $g(x)=x^\alpha$ will be adopted, and the uncertainty due to questioning the model will be ignored. The data for Italy places $\alpha$ around $0.4$ and the data for USA places $\alpha$ around $0.55$.

\bigskip

\noindent {\bf The RTT method to solve differential equations with data subject to noise}. The relatively novel theoretical contribution of this report is the RTT method to mimic systems of deterministic differential equations by stochastic counterparts, simpler than, and conceptually and practically different from, the more common stochastic differential equations based on Diffusion processes.

Parameter estimation ($\beta, \gamma, K, \alpha$) will be performed by the Random Time Transformation (RTT) method developed by Bassan, Marcus, Meilijson and Talpaz \cite{Bassanetal}, motivated by the notion of Skew Product in Ergodic Theory. Unlike Diffusion methods that place noise vertically, the RTT method adopts the solution to the deterministic system of differential equations, but considers it as evaluated at a random time process that advances on the average like chronological time. This random time is modelled in practice as a Gaussian process (Brownian Motion or Ornstein-Uhlenbeck) and this provides a likelihood model for the estimation of parameters inherent in the SIR (or otherwise) system of ODE. The differential terms ($g(I(t)) \max(0,1 - X(t)/K)$ and $I(t)$) in equations ({\ref{DEforX},\ref{DEforR}) identify the Jacobian term in the likelihood function, for the application of MLE, including both point estimates and standard errors. As will be seen, for fixed $K$ and $\alpha$ the RTT method identifies $\beta$ and $\gamma$ directly, without reference to the Gaussian part. The likelihood model to be described thus provides a profile likelihood in terms of $K$ and $\alpha$ only.

\bigskip

Let the empirical data consist of $(X_1,R_1), (X_2,R_2), \dots, (X_n, R_n)$, from which the infected case totals $Y_j=X_j-R_j$ can be inferred. As expressed above, the value of $\beta$ at time $j$ is noisily provided by $A_j={{X_j-X_{j-1}} \over {Y_{j-1}(1-{{X_j} \over N})}}$ or by a local smooth regression.

The plan to be pursued is to solve the system above of ODE globally, as adequately as possible, and if this plan succeeds and produces smooth, calculable, functions $(x,r,i)$ that tightly fit the empirical data, we will have a prediction of the maximal future damage $K$ that $X$ will sustain, as well as a smooth function $A(t)={{{dx(t)} \over {dt}} \over {y(t)}}$ that replaces and extrapolates the noisy local growth ratio $A_j$. This function may better pinpoint when did $A$ cross below $\gamma$, or express a predicted transition time below $\gamma$, or warn that such a transition is not due to happen in the foreseeable future.

\bigskip

With this in mind, here is a detailed description of the RTT method. No attempt will be made to solve the SIR system analytically. Instead, a small increment of time $\delta={1 \over M}$ is set, and the ODE is solved numerically as a difference equation. Interpreting as time the indices of the empirical data, $M=100$ is a reasonable choice.

Fix the parameters $\beta, \gamma, K$ and the function $g$, initiate functions $x$ and $r$ as $X_1$ and $R_1$ respectively, initiate $i$ as $X_1-R_1$ and proceed with the definition for $j \ge 2$
\begin{eqnarray}
x(j)&=&x(j-1)+\beta g(i(j-1))\max(0,K-x(i-1)) \delta \nonumber \\
r(j)&=&r(j-1)+\gamma i(j-1) \delta \nonumber \\
i(j)&=&max(0,x(j)-r(j)) \label{thesolution}
\end{eqnarray}

Regular Least Squares essentially estimate parameters by minimizing a sum of squares of the vertical errors $X(i)-x(M i), R(i)-r(M i), I(i)-i(M i)$. The RTT idea works instead with horizontal errors:

\bigskip

Define the {\em random time trajectory} as starting as $T_1(1)=1, T_2(1)=1$. For $m=2,3,\cdots,n$, let $T_1(m)$ be the smallest $j$ for which $x(j) \ge X_j$ and let $T_2(m)$ be the smallest $j$ for which $r(j) \ge R_j$.

Now solve for $\beta$ and $\gamma$ so that $T_1(n)=T_2(n)=n$. That is, incremental time has average $1$ in both equations. Define $\Delta_1(m)=T_1(m+1)-T_1(m)$ and $\Delta_2(m)=T_2(m+1)-T_2(m)$, for $m=1, 2, \cdots,m-1$ as the (mean-$1$) increments of the $T_1$ amd $T_2$ processes.

\bigskip

If population size $N$ was substituted for $K$ and the identity function for $g$, work is over. For quality-of-fit sanity check, plot the $(x,r,i)$ solution with the $(X,R,I)$ data, that agree at both time endpoints.

Alternatively, consider $K$ and $\alpha$ as free parameters, to be estimated from the data. 

\bigskip

\noindent {\bf The likelihood function induced by the RTT method}. View the incremental times \linebreak $(\Delta_1(m),\Delta_2(m))$ as observations from a bivariate mean-zero Gaussian distribution, and let $\Sigma$ be their empirical covariance matrix. Up to a multiplicative constant, the normal density evaluated at these data is $(\det(\Sigma))^{-{{n-1} \over 2}}$.

To obtain the likelihood function, the above density must be multiplied by the Jacobian of the transformation. This can be easily seen to be $1$ over the product over the sample of the differential terms $\beta g(X_m-R_m)\max(0,1-X_m/K)$ and $\gamma (X_m-R_m)$, perhaps evaluated at
${{X_m+X_{m-1}} \over 2}$ and ${{R_m+R_{m-1}} \over 2}$ instead of $X_m$ and $R_m$.

\bigskip

The parameters $K$ and $\alpha$ are MLE-estimated by maximizing the logarithm of this profile likelihood function, and their standard errors (and correlation coefficient, if needed) are estimated as usual, via the empirical Fisher information.

\begin{itemize}

\item Point and interval estimates for $K$ and $\alpha$ will be calculated.

\item Point and interval estimates will be evaluated to validate the data at the last day, based on the analysis of the data ending $3,6,9,12,15$ days before the last day.

\item The date with maximal number of infected cases and the date of transition of $\beta$ below $1$ will be reported, with confidence intervals based on $K$-values at the endpoints of the confidence interval for $K$.

\end{itemize}

\section{Covid19 analysis} \label{Covid19}

As will be illustrated,

\begin{itemize}

\item Italian and USA 2020 data until May 12nd provide a totally inacceptable fit for $\alpha=1$ and $K=N$.

\item Italian data provide good fit for both $\alpha=1$ and $\alpha={1 \over 2}$, provided $K$ is left free, in which case it is some $20\%$ higher than the May 12 value of $X$.

\item USA data with $\alpha=1$ provide very inadequate fit for all $K$, and at the MLE of $K$, it is predicted that the number of infected cases $I$ should have reached a maximum on May 8.

\item USA data with $\alpha={1 \over 2}$ provide very good fit, estimating $K$ as roughly twice the $X$-value on May 2, and predicting that $I$ will reach maximal value around May 25. It predicted an increase in $X$ from May 2 to May 9 by some $80000$ new affected cases, and the actual value was $102000$.

\end{itemize}

Graphs are provided next.

\section{A remark on incremental new cases} \label{More}

In principle, the increments $X(i)-X(i-1)$ should be independent, Poisson distributed with some local mean, provided by the differential equations. Poisson random variables $Z$, positive and with variance equal to the mean, should display when this mean is at least $5$ or so, the remarkable property that $\sqrt{Z}$ has standard deviation very close to, and fast converging to ${1 \over 2}$. Whether or not this theoretical fact is satisfied empirically by the $F(i)=\sqrt{max(1,X(i)-X(i-1))}$ data can be checked without reference to the differential equations. Simply, choose a window size $WS$ such as $5$ or $10$, do for each $i$ linear regression of $F(i-WS:i+WS)$ on time $(i-WS:i+WS)$, and record the averages and slopes or correlation coefficients of these lines as well as the standard deviations $\hat{\sigma_i}$ of the residuals. These standard deviations are supposed to estimate ${1 \over 2}$, or else assist in data diagnosis or identify a batch-size of arrivals. Italy and USA display weekly seasonality in which significantly less cases are reported on weekends, and delayed reporting could be more the rule than the exception. In any case, the empirical standard deviations $\hat{\sigma_i}$ are so much bigger than ${1 \over 2}$ that probabilistic modelling methods based on the Poisson hypothesis are not justified.

Graphs are provided next.



\section*{Acknowledgements}

Thanks are due to Ilan Eshel from prompting this study and to Eytan Ruppin, Amit Huppert and David Steinberg for helpful suggestions.

\baselineskip= 28pt

\begin{thebibliography}{99}

\bibitem{Bassanetal} Bassan, B., Marcus, R., Meilijson, I. and Talpaz, H. (1997). Parameter erstimation in differential equations, using random time transformations. {\em Journal of the Italian Statistical Society}, {\bf 6}, 177--199.

\bibitem{AAA} Grenfell, B.T., Bj{\o}rnstad, O.N. and Filkenst\"{a}dt, B. A. (2002) Dynamics of Measlres epidemics: scaling, noise, determinism and predictability with the TSIR model. {\em Ecological Monographs}, {\bf 72(2)}, 185-–202.


\bibitem{BBB}
	
\end{thebibliography}

\end{document}
