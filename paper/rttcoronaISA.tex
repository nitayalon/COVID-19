%\documentclass[12p]{article}
%\usepackage{a4paper,makeidx}
%\usepackage[dvips]{graphics}
\documentclass{article}
\usepackage{graphicx}
\usepackage{subfig}
\usepackage{multirow}

%\documentclass[smallextended]{svjour3}
%\usepackage[dvips]{graphicx}
%\usepackage{subfig}
%%%%%%%%%%%%%%%%%%%%%%%%%%%%%%%%%%%%%%%%%%%%%%%%%%%%%%%%%%%%%%%%%%%%%%%%%%%%%%%%%%%%%%%%%%%%%%%%%%%%%%%%%%%%%%%%%%%%%%%%%%%%

\righthyphenmin=55
\newlength{\gnat}
\newlength{\figheight}
\newlength{\figwidth}
\setlength{\textwidth}{6in}
\setlength{\evensidemargin}{0.2in}
\setlength{\oddsidemargin}{0.2in}
\setlength{\topmargin}{0.0in}
\setlength{\textheight}{9in}
\setlength{\headsep}{10pt}
\setlength{\columnsep}{0.375in}

\newtheorem{theorem}{Theorem}
\newtheorem{proposition}{Proposition}
\newtheorem{lemma}{Lemma}
\newtheorem{corollary}{Corollary}
\newtheorem{definition}{Definition}
\newtheorem{remark}{Remark}
\newtheorem{claim}{Claim}

\def\IR{\Bbb R}
\def\tF{{\tilde F}}
\def\tG{{\tilde G}}
\def\tE{{\tilde E}}
\def\hF{{\hat F}}
\def\hG{{\hat G}}
\def\hE{{\hat E}}
\def\oF{{\overline{F}}}
\def\oG{{\overline{G}}}
\def\CC{{\cal C}}
\def\uF{{\underline{F}}}
\def\uG{{\underline{G}}}
\def\oC{{\overline{C}}}
\def\uC{{\underline{C}}}
\baselineskip= 20pt
%\input{tcilatex}

\begin{document}

\title {Random time transformation analysis of the Covid19 Pandemic 2020}

\author {
Nitay Alon
\\
Isaac Meilijson
\\
{\em Tel-Aviv University} \\
\\
\\
Israel Statistical Association \\
Virtual workshop on statistical aspects of the SARS-CoV-2 pandemic \\
May 21, 2020}


%\titlerunning{Covid19 2020}
%\authorrunning{Alon and Meilijson}

\date{}

\maketitle

\pagenumbering{arabic}

\newpage
%\vspace{1cm}

%\noindent \hrulefill \hspace{12cm}

%\baselineskip= 20pt

\huge

%\section{Introduction} \label{introduction}

\begin{center} 
{\bf SIR epidemiological model}
\end{center} 

\bigskip

\noindent $S(t) = K - X(t)$ susceptible cases at time $t$ 

\noindent $X(t)$ affected cases, $K$ usually population size $N$

\noindent $R(t)$ removed cases (dead $+$ recovered)

\noindent $I(t)=S(t)-R(t)$ infected cases

%The common formulation of the SIR (Susceptible, Infected, Removed) epidemiological model is the variant of the system of ODE
\begin{eqnarray}
dX(t) & = & \beta(t) g(I(t)) (1 - X(t)/K) dt \nonumber \\
dR(t) & = & \gamma I(t) \nonumber dt \\
I(t) & = & X(t)-R(t) \nonumber
\end{eqnarray}
where $K=N$ and $g(x)=x$. Or else.

\newpage 

\noindent $\beta(t)$ smoothly estimated from $X$-increments and $I$.

\bigskip

\noindent Ratio $R_0={{\beta(t)} \over \gamma}$ of primary importance, transition from above to below $1$ $=$ epidemic spreading or dwindling $=$ infected cases $I$ increase or decrease with time.

\bigskip

\noindent At early stages of epidemic, substitution of $K$ by population size $N$ or $\infty$, has similar effects. 

\bigskip

\noindent Will illustrate on Covid19 2020 that if $K$ free parameter, SIR system accepts almost exact solution with $\beta$ (not only $\gamma$) constant, $K$ is asymptote of $X$, {\bf if conditions prevalent at monitoring stage stay in effect}

\newpage

\noindent Linear appearance of susceptibles in $dX$ and infected in $dR$ quite straightforward under stationary regime, but role of $I(t)$ in $dX$ needs attention. Effective vicinity of susceptibles need not be entire infected cohort, and asymptote of affected cases need not be population size $N$.

\bigskip

\noindent "As if known" $g$ may provide unjustifiably accurate estimates of asymptote $K$. $g(x)=x^\alpha$ for some \linebreak $\alpha<1$ repeatedly suggested (Bj{\o}rnstad, Finkenst\"{a}dt and Grenfell and references therein).

\newpage 

\noindent $\alpha$ estimated with $K$ (and $(\beta, \gamma)$). Even if $\hat{K}$ and $\hat{\alpha}$ were uncorrelated (same marginal and conditional SE of $\hat{K}$), $K$ may be sensitive to choice of $\alpha$:  Ill-advised to set a compromise value for all countries.

\bigskip

\noindent For now, $K$ estimated by MLE $\hat{K}$ country by country. Empirical Bayes planned.

\bigskip

\noindent Model $g(x)=x^\alpha$ adopted, uncertainty due to \linebreak questioning model ignored.

%\noindent Not so: $\hat{K}$ and $\hat{\alpha}$ strongly negatively correlated and compensate for each other. Profile likelihood of $K$ with $\alpha$ nuisance gives FISHER much smaller than with $\alpha$ fixed.

\newpage

\noindent Covid19 2020 indicates country by country that \linebreak measures taken March to May 2020 induced stationary behavior, with $K$ strongly significantly finite, not exceeding $3$ times number of affected cases in mid May 2020.

\bigskip

\noindent Italy Data $\alpha \approx 0.4$, USA data $\alpha \approx 0.55$ 


\newpage

\noindent How to solve SIR system and estimate parameters? Least Squares, Diffusion methods with Fokker-Planck transition density. Tough, especially with multiple equations.

\bigskip

\noindent Purpose of talk to introduce RTT of Bassan, Marcus, M., Talpaz (1997), Volcani Institute viral spread in plants. Based on Skew Product idea in Ergodic Theory.

\newpage

\noindent {\bf RTT method to solve ODE with noisy data} 

\bigskip

\noindent Mimic and use solution to the system of deterministic DE: simpler, conceptually and practically different from more common SDE based on Diffusion processes.

\bigskip

\noindent Diffusion methods place noise vertically, RTT adopts solution to deterministic ODE, but evaluated at a random time process that advances on the average like chronological time. So noise is horizontal. 

\newpage

\noindent Random time is modelled in practice as if it was Brownian Motion (or Ornstein-Uhlenbeck). 

\bigskip

\noindent This provides a likelihood model for estimation of parameters inherent in SIR (or otherwise) ODE. 

\bigskip

\noindent The terms $g(I(t)) (1 - X(t)/K)$ and $I(t)$ identify the Jacobian term in the likelihood function, which plays a role of penalized Least Squares favoring \linebreak smoothness.

\newpage

\noindent For fixed $K$ and $\alpha$, RTT identifies $\beta$ and $\gamma$ directly, without reference to the Gaussian part. 

\noindent The likelihood model to be described thus provides a profile likelihood in terms of $K$ and $\alpha$ only.

\bigskip

\noindent Empirical data $(X_1,R_1), (X_2,R_2), \dots, (X_n, R_n)$, with infected case totals $Y_j=X_j-R_j$. 

\bigskip

\noindent For fixed $K, \alpha, \beta, \gamma$ write ODE deterministic solution $(x,r)$ numerically, with initial values $X_1$ and $R_1$ and time scale $M$ (say, $100$) per day. Let $\delta={1 \over M}$.   

\newpage

\begin{eqnarray}
x(j)&=&x(j-1)+\beta g(i(j-1))(1-{{x(i-1)} \over K}) \delta \nonumber \\
r(j)&=&r(j-1)+\gamma i(j-1) \delta \nonumber \\
i(j)&=&x(j)-r(j)\nonumber
\end{eqnarray}

\bigskip

\noindent Define the {\em random time trajectory} as starting as $T_1(1)=1, T_2(1)=1$. For $m \ge 2$,

\bigskip

\noindent $T_1(m)$ is smallest $j$ with $x(j) \ge X_j$ and $T_2(m)$ smallest $j$ with $r(j) \ge R_j$.

\bigskip

\noindent Solve for $\beta$ and $\gamma$ so that $T_1(n)=T_2(n)=n$. 

\noindent Incremental time has average $1$ in both equations. 

\newpage

\noindent The centered increments of the $T_1$ and $T_2$ processes

\bigskip

\noindent $\Delta_1(m)=T_1(m+1)-T_1(m)-1$ 

\noindent $\Delta_2(m)=T_2(m+1)-T_2(m)-1$

\noindent viewed as observations from a mean-zero BVN and emp cov matrix $\Sigma$ estimates the covariance.
Normal density proportional to $(\det(\Sigma))^{-{{n-1} \over 2}}$.

Jacobian is $1$ over product over sample of differential terms $\beta g(X_m-R_m)(1-X_m/K)$ and $\gamma (X_m-R_m)$.

\bigskip

\noindent $K$ and $\alpha$ MLE-estimated by this profile likelihood, standard errors from empirical Fisher information.

\newpage

\section*{Acknowledgements}

Thanks are due to Ilan Eshel from prompting this study and to Eytan Ruppin, Amit Huppert and David Steinberg for helpful suggestions.

\baselineskip= 28pt

\begin{thebibliography}{99}

\bibitem{Bassanetal} Bassan, B., Marcus, R., Meilijson, I. and Talpaz, H. (1997). Parameter erstimation in differential equations, using random time transformations. {\em Journal of the Italian Statistical Society}, {\bf 6}, 177--199.

\bibitem{AAA} Grenfell, B.T., Bj{\o}rnstad, O.N. and Filkenst\"{a}dt, B. A. (2002) Dynamics of Measlres epidemics: scaling, noise, determinism and predictability with the TSIR model. {\em Ecological Monographs}, {\bf 72(2)}, 185-–202.

\bibitem{BBB}
	
\end{thebibliography}

\end{document}
